\documentclass{article}

% If you're new to LaTeX, here's some short tutorials:
% https://www.overleaf.com/learn/latex/Learn_LaTeX_in_30_minutes
% https://en.wikibooks.org/wiki/LaTeX/Basics

% Formatting
\usepackage[utf8]{inputenc}
\usepackage[margin=1in]{geometry}
\usepackage[titletoc,title]{appendix}
\usepackage{booktabs}
\usepackage{stfloats}

% Math
% https://www.overleaf.com/learn/latex/Mathematical_expressions
% https://en.wikibooks.org/wiki/LaTeX/Mathematics
\usepackage{amsmath,amsfonts,amssymb,mathtools}

% Images
% https://www.overleaf.com/learn/latex/Inserting_Images
% https://en.wikibooks.org/wiki/LaTeX/Floats,_Figures_and_Captions
\usepackage{graphicx,float}
\usepackage[utf8]{inputenc}

% Tables
% https://www.overleaf.com/learn/latex/Tables
% https://en.wikibooks.org/wiki/LaTeX/Tables

% Algorithms
% https://www.overleaf.com/learn/latex/algorithms
% https://en.wikibooks.org/wiki/LaTeX/Algorithms
\usepackage[ruled,vlined]{algorithm2e}
\usepackage{algorithmic}

% Code syntax highlighting
% https://www.overleaf.com/learn/latex/Code_Highlighting_with_minted
\usepackage{minted}
\usemintedstyle{borland}

% Title content
\title{ECON 5253 Problem Set 9}
\author{Ahmed Chaudhry}
\date{\today}

\begin{document}
\maketitle

\section{Question No. 7}
The original housing dataset has 15 variables in which 14 are X variables. However, in housing\_training has a total of 75 variables, so the X variable difference is of about 60 variables.

\section{Question No. 8}
Optimal $\lambda$ (penalty parameter) = 0.00356\\
In-sample RMSE = 0.220 \\
Out-sample RMSE = 0.219 \\

\section{Question No. 9}
Optimal $\lambda$ (penalty parameter) = 0.0233\\
Out-sample RMSE = 0.192 \\

\section{Question No. 10}
We cannot have a unique solution when predictors are more than observations. However, if we use RIDGE or LASSO regression, this problem can be catered to.

Overall, I think for both LASSO and RIDGE we have somewhat handled the problem of Bias-Variance Trade-off.
\end{document}

