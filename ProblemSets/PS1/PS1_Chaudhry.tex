\documentclass{article}

% If you're new to LaTeX, here's some short tutorials:
% https://www.overleaf.com/learn/latex/Learn_LaTeX_in_30_minutes
% https://en.wikibooks.org/wiki/LaTeX/Basics

% Formatting
\usepackage[utf8]{inputenc}
\usepackage[margin=1in]{geometry}
\usepackage[titletoc,title]{appendix}

% Math
% https://www.overleaf.com/learn/latex/Mathematical_expressions
% https://en.wikibooks.org/wiki/LaTeX/Mathematics
\usepackage{amsmath,amsfonts,amssymb,mathtools}

% Images
% https://www.overleaf.com/learn/latex/Inserting_Images
% https://en.wikibooks.org/wiki/LaTeX/Floats,_Figures_and_Captions
\usepackage{graphicx,float}

% Tables
% https://www.overleaf.com/learn/latex/Tables
% https://en.wikibooks.org/wiki/LaTeX/Tables

% Algorithms
% https://www.overleaf.com/learn/latex/algorithms
% https://en.wikibooks.org/wiki/LaTeX/Algorithms
\usepackage[ruled,vlined]{algorithm2e}
\usepackage{algorithmic}

% Code syntax highlighting
% https://www.overleaf.com/learn/latex/Code_Highlighting_with_minted
\usepackage{minted}
\usemintedstyle{borland}

% Title content
\title{ECON 5253 Problem Set 1}
\author{Ahmed Chaudhry}
\date{\today}

\begin{document}
\maketitle
\section{My Research Interests}
I am interested in studying the interaction between politics and economics, i.e., public economics and political economy. As a student of Economics, I know that macroeconomic stability is imperative to bring the economic development of a country. But I am also cognizant of the fact that politics, state, society and its institutions play a central role in determining the performance of an economy. Thus, these factors cannot be divorced from economic analysis. Therefore, being the study of the economy as a whole, Economics, I believe, cannot be isolated from social and political realities of a country, i.e., political economy.

What makes me curious, in particular, is how electoral outcomes and political decisions affect economic policies. It is crucial to examine this relationship because political parties and politicians, in their pursuit of winning the elections, are susceptible to making decisions that would benefit them only, i.e., they seek to maximize their votes. For example, given limited resources, if incumbents are faced with a decision either to build state-of-the-art infrastructure, like an IT park, or establish a new research university. The incumbents would most likely spend public resources on a project that would maximize their share of votes.

I am interested in learning data science techniques because handling and analyzing Big Data is a skill which would be useful for my own research. In particular, I am interested in analyzing the Tweets (Twitter) by various politicians from my home country, Pakistan, in order to understand what promises they make in their manifestos vs what they say on twitter vs how actually they employ public expenditures when the come into power. So, I want to look into the "politics of promises" in Pakistan's case. At this point, this is just a prospective research project that I want to work on during this course, I am not sure if I end up doing it or not.

After the completion of my Ph.D., I would like to work as an academic.

\section{Equations}
$$a^2 + b^2 = c^2$$
\end{document}
